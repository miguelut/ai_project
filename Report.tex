\documentclass[a4paper,11pt]{report}
\usepackage[T1]{fontenc}
\usepackage[utf8]{inputenc}
\usepackage{lmodern}
\usepackage{setspace}
\onehalfspacing

\title{AI: An Overview of Uninformed Searching and Applications}
\author{Dylan Thompson \and Michael M. Wright}

\begin{document}

\maketitle
\tableofcontents

\begin{abstract}
This paper begins with a broad overview of uninformed search algorithms in the 
field of Artificial Intelligence.  While search is useful for a variety of toy 
problems such as the Wolf-Goat-Cabbage problem and the Missionaries and
Cannibals problem, this paper will discuss some of the real-world problems
for which uninformed searching is an acceptable solution.  The paper will then 
analyze some of the different search algorithms as they relate to solving a 
Rubik's cube.

\textbf{Don't forget to add your conclusions!}

\end{abstract}

\chapter{Searching}
A small portion of solvable problems can be modeled in computers simply by
mapping the entire set of conditions to corresponding actions.  However, for
some problems this mapping is impractical if not impossible given the
constraints of our current technology.  In situations like these it becomes
imperative to build a \textbf{goal-based} agent.  Goal-based agents can, in
effect, consider future actions and evaluate the effect of those future
actions on reaching the desired solution.\cite[p.~64]{norvig}  The two main
subfields of goal-based agents are search and planning.\cite{wikiAgent} The
field of searching can be further whittled down into those algorithms that
utilize uninformed algorithms and those that utilize informed algorithms.
Later in the chapter we will look at the uninformed searching algorithms in
use today.  First, however, we must consider the criteria a problem must meet
before we can use a search agent to solve that problem.  It turns out that a
problem needs to be \textbf{well-defined} prior to unleashing a searching
agent to solve the problem.

\section{Defining the Problem}


\begin{thebibliography}{9}

  \bibitem{norvig}
    Stuart Russel and Peter Norvig,
    \textit{Artificial Intelligence: A Modern Approach}.
    Prentice Hall, Upper Saddle River,
    3rd Edition,
    2010.

  \bibitem{wikiAgent}
    Intelligent Agent. 21 April, 2013.  
    In \textit{Wikipedia}.  Retrieved April 25, 2013,
    from http://en.wikipedia.org/wiki/intelligent\_agent

\end{thebibliography}

\end{document}
